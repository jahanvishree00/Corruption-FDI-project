\documentclass{article}

% Language setting
% Replace `english' with e.g. `spanish' to change the document language
\usepackage[english]{babel}

% Set page size and margins
% Replace `letterpaper' with `a4paper' for UK/EU standard size
\usepackage[letterpaper,top=2cm,bottom=2cm,left=3cm,right=3cm,marginparwidth=1.75cm]{geometry}

% Useful packages
\usepackage{amsmath}
\usepackage{graphicx}
\usepackage[colorlinks=true, allcolors=blue]{hyperref}
\usepackage{multirow}
\usepackage[utf8]{inputenc}
\usepackage{pgfplots}
\usepackage{pgfplotstable}

\pgfplotsset{compat=1.14}

\title{Impact of Corruption on Foreign Direct Investment: A Comparative Study of India, Japan,  and Saudi Arabia}
\author{Jahanvi Shree}

\begin{document}
\maketitle

\begin{abstract}

\end{abstract}

\section{Introduction}
Foreign Direct Investment(FDI) is essential  for job creation, economic growth,and technological advancement in developing nations. The governance standards and institutional quality of a nation have become important factors in determining investment decisions as countries vie for the attention of international investors. By raising operational risks, decreasing transparency, and raising the cost of doing business, corruption—which is defined as the misuse of entrusted power for personal benefit—poses a serious obstacle to foreign investment.


This study examines the connection between levels of corruption and foreign direct investment inflows, with a particular emphasis on India and comparative analysis from nations like Saudi Arabia and Japan. Even though India has continuously worked to streamline its bureaucratic red tape and enhance its regulatory environment, ongoing corruption problems might be making the country less appealing to foreign investors. Conversely, nations with significantly lower (Japan) or differently structured (Saudi Arabia) corruption profiles exhibit divergent trends in investment inflow, providing insightful background information for analysis.


This study investigates the relationship between FDI inflow volumes and Corruption Perceptions Index (CPI) scores using annual data from 2008 to 2023. It also assesses whether shifts in corruption levels have a major impact on investor's behaviour in various institutional contexts. By doing this, the project hopes to advance knowledge of how the quality of governance, in particular the fight against corruption, can influence the investment climate in both developed and developing nations.


\section{Literature Review}

In discussions about the global economy, it is becoming more and more crucial to comprehend the connection between corruption and foreign direct investment (FDI). The impact of governance quality—in particular, corruption—on the choices of foreign investors has been the subject of much research by academics and institutions. The argument that corruption discourages FDI inflows by raising business uncertainty, inflating transaction costs, and undermining investor confidence is supported by a substantial body of literature.{1} However, the political system, economic structure, and institutional resilience of a nation all affect how much of an impact this has. Government incentives, natural resource wealth, and reforms all have an impact on the complex relationships between corruption levels and foreign direct investment (FDI) in countries like Saudi Arabia and India. In contrast, countries like Japan exhibit a robust institutional framework with low levels of corruption.


\begin{thebibliography}{99}

\bibitem{1. Habibi, M., & Zurawicki, L.(2002).} 
Habibi M. & Zurawicki L.(2002). \textit{Corruption and foreign direct investment: An empirical analysis. Journal of International Business Studies, 33(2), 291-307}.  
Retrieved from \url{https://doi.org/10.1057/palgrave.jibs.8491013}




\end{thebibliography}


\end{document}